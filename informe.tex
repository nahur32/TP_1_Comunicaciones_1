\documentclass{article}

% Configuración de la codificación de entrada
\usepackage[utf8]{inputenc}

% Configuración de la codificación de salida
\usepackage[T1]{fontenc}

% Paquete para el idioma y comillas
\usepackage[spanish,es-tabla]{babel}
\usepackage{csquotes}

\usepackage{hyphenat}
\usepackage{microtype}

\hyphenation{ex-am-ple hy-phen-a-tion}
\hyphenpenalty=500
\tolerance=1000

% Tamaño de la página (A4) y márgenes
\usepackage[a4paper,top=2cm,bottom=2cm,left=3cm,right=3cm,marginparwidth=1.75cm]{geometry}

% Otros packages
\usepackage{amsmath}
\usepackage{graphicx}
\usepackage[colorlinks=true, allcolors=blue]{hyperref}
\usepackage[utf8]{inputenc}

\usepackage{hyperref}
\usepackage{amsmath}
\usepackage[usenames]{color}
\usepackage{amsfonts}
\usepackage{amssymb}
\usepackage{graphicx}
\usepackage{caption}
\usepackage{enumerate} 
\usepackage{siunitx}
\usepackage{upgreek}
\usepackage{epsfig}
\usepackage{multirow}
\usepackage{colortbl}
\usepackage{xspace}    

% para las tablas
\usepackage{multicol,multirow, array} 
\usepackage[table,xcdraw]{xcolor}
\usepackage[table]{xcolor}
\captionsetup{justification=centerlast,labelfont=bf,font=sf}
\usepackage{subfigure}
\usepackage[T1]{fontenc}
\usepackage{fourier}
\usepackage{fancyhdr}
\usepackage{float} 
\usepackage{steinmetz}
\setcounter{equation}{0}
\usepackage{biblatex}

% para agregar la carpeta de las imagenes
\graphicspath{{./nombre de la carpeta/}}

\begin{document}
	
	% --- Carátula ---
	\begin{titlepage}	  
		\centering
		
		% --- Logo UNSL ---	
		\begin{figure}
			\centering
			\includegraphics[width=0.15\linewidth]{logo-unsl.jpg}
		\end{figure}    
		
		% --- Datos de la asignatura ---
		{\scshape\LARGE Universidad Nacional de San Luis\par}
		{\scshape Facultad de Ciencias Físico Matemáticas y Naturales\par}
		{\scshape Ingeniería Electrónica con O.S.D.\par}
		\bigskip 
		\bigskip 
		\bigskip 
		
		\Large \textbf {Asignatura:\\} 
		\LARGE Diseño de Sistemas Digitales	
		\bigskip 
		\bigskip 
		\bigskip
		\bigskip
		
		% --- Datos del informe / TP ---
		\LARGE \textbf {Informe de laboratorio\\}
		\LARGE \textbf {TP N° 2: Diseño de lógica combinacional con VHDL}
		
		\bigskip
		\bigskip
		\bigskip
		
		% --- Datos del alumno ---  
		\LARGE \textbf {Estudiante:\\} 
		\LARGE Nombre y apellido del alumno\\
		\bigskip
		\Large \textbf {DNI N°:\\} 
		\Large Número de DNI	
		
		\bigskip
		\bigskip
		\bigskip
		\bigskip
		\bigskip
		\bigskip
		
		% --- Datos de los profesores ---  
		\Large \textbf {Profesores responsables:\\} 
		\Large Dr. Víctor Yelpo \\ Ing. Ivana Trento
		
		\bigskip
		\bigskip
		\bigskip
		\bigskip
		\bigskip
		\bigskip
		\bigskip
		\bigskip
		\bigskip
		
		% --- Año ---  
		\Large \textbf {Año:\\} 
		\Large 2024	
	\end{titlepage}
	\newpage
	
	% --- Fin de la carátula --- 
	
	
	
	% --- Comienzo del informe --- 
	
	\section*{Parte I: Conceptos teóricos}

\subsection*{1. Sistemas de Comunicación}

\textbf{1.1)} Dibuje el diagrama de bloques de un sistema de comunicación genérico e identifique cada una de sus partes principales. Explique brevemente la función de cada bloque.

\textbf{1.2)} ¿Cómo se clasifican los sistemas de comunicación (sdc) según los siguientes criterios? De ejemplos de cada caso.
\begin{itemize}
    \item Por el tipo de señal
    \item Por la dirección de transmisión
    \item Por la sincronización
    \item Por el medio de transmisión
\end{itemize}

\end{document}