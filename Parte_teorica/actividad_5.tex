\subsection*{5. Representación de Señales Pasabanda}

\noindent \textbf{5.1)} Explique la diferencia entre señales de banda base y señales pasabanda. \par
\bigskip

    \noindent Una forma de onda de banda base tiene una magnitud espectral diferente de
    cero para las frecuencias alrededor del origen (es decir, f = 0) y es despreciable en cualquier
    otro caso portadora.\par
    \bigskip
    \noindent Una forma de onda pasabanda tiene una magnitud espectral diferente de cero
    para las frecuencias en cierta banda concentrada alrededor de una frecuencia f =+-fc, donde
    fc  0. La magnitud espectral es despreciable en cualquier otro caso. A la frecuencia fc se le
    llama frecuencia portadora.


%---------------------------------------------------------------------------------
\bigskip

\noindent \textbf{5.2)} Describa la representación de envolvente compleja de una señal pasabanda. ¿Por qué es útil esta representación? \par
\bigskip

\noindent Una señal pasabanda real \( s(t) \) con frecuencia portadora \( f_c \) puede expresarse en términos de su envolvente compleja \( \tilde{s}(t) \) de la siguiente manera:\par

\noindent \textbf{Forma general:}
\[
s(t) = \text{Re} \left\{ \tilde{s}(t) \, e^{j 2\pi f_c t} \right\}
\]

\noindent \textbf{Descomposición en componentes:}\par
\noindent La envolvente compleja \( \tilde{s}(t) \) es una señal compleja de banda base que se escribe como:
\[
\tilde{s}(t) = s_I(t) + j s_Q(t)
\]
donde:
\begin{itemize}
    \item \( s_I(t) \): \textbf{Componente en fase} (parte real),
    \item \( s_Q(t) \): \textbf{Componente en cuadratura} (parte imaginaria).
\end{itemize}

\noindent \textbf{Expresión alternativa de \( s(t) \):}\par
\noindent Sustituyendo \( \tilde{s}(t) \) en la forma general:
\[
s(t) = s_I(t) \cos(2\pi f_c t) - s_Q(t) \sin(2\pi f_c t)
\]

\noindent \textbf{Forma polar de la envolvente compleja:}
\[
\tilde{s}(t) = a(t) \, e^{j \phi(t)}
\]
donde:
\begin{itemize}
    \item \( a(t) = \sqrt{s_I^2(t) + s_Q^2(t)} \): \textbf{Envolvente real} (amplitud instantánea),
    \item \( \phi(t) = \tan^{-1} \left( \dfrac{s_Q(t)}{s_I(t)} \right) \): \textbf{Fase instantánea}.
\end{itemize}
\noindent Entonces, la señal pasabanda se expresa como:
\[
s(t) = a(t) \cos \left( 2\pi f_c t + \phi(t) \right)
\]

\noindent Esta representación es fundamental porque permite analizar y procesar señales pasabanda (de alta frecuencia) utilizando herramientas de banda base (baja frecuencia), lo que simplifica enormemente el estudio de su comportamiento espectral y temporal, reduce la complejidad computacional en simulaciones al requerir menores tasas de muestreo, y facilita la aplicación de operaciones como filtrado, convolución o transformaciones al trabajar con señales de variación lenta, manteniendo la generalidad para todo tipo de modulaciones.
%---------------------------------------------------------------------------------
\bigskip

\noindent \textbf{5.3)} Para una señal modulada \(s(t) = A(t)\cos[\omega_c t + \phi(t)]\), identifique: \par

\noindent a) La componente en fase.\par
\bigskip

\begin{align*}
s(t) &= A(t) \cos[\omega_c t + \phi(t)] \\
     &= A(t) \cos(\omega_c t) \cos \phi(t) - A(t) \sin(\omega_c t) \sin \phi(t)
\end{align*}
Comparando con la forma canónica:

\[
s(t) = s_I(t) \cos(\omega_c t) - s_Q(t) \sin(\omega_c t)
\]

Se identifica:

\[
s_I(t) = A(t) \cos \phi(t)
\]
\bigskip

\noindent b) La componente en cuadratura. \par
\bigskip
De la misma expansión:

\[
s(t) = s_I(t) \cos(\omega_c t) - s_Q(t) \sin(\omega_c t)
\]

Se identifica:

\[
s_Q(t) = A(t) \sin \phi(t)
\]
\bigskip

\noindent c)La envolvente compleja.\par
\bigskip

La envolvente compleja se define como:

\[
\tilde{s}(t) = s_I(t) + j s_Q(t)
\]

Sustituyendo los valores obtenidos:

\[
\tilde{s}(t) = A(t) \cos \phi(t) + j A(t) \sin \phi(t) = A(t) e^{j\phi(t)}
\]

