\subsection*{2. Representación Fasorial y Análisis Frecuencial}

\textbf{2.1)} Dada la señal \( x(t) = 5\cos(2\pi \cdot 1000t + \pi/4) + 3\sin(2\pi \cdot 1500t - \pi/6) \):
\begin{enumerate}[label=\alph*)]
    \item Exprese cada componente en forma fasorial
    \item Grafique el diagrama fasorial
    \item Encuentre la representación exponencial compleja
\end{enumerate}

\textbf{2.2)} Explique el concepto de fasor rotante y su utilidad en el análisis de señales sinusoidales.
