\subsection*{3. Series y Transformada de Fourier}

\textbf{3.1)} Enuncie las condiciones de Dirichlet para la existencia de la Serie de Fourier.

\textbf{3.2)} Para una señal periódica cuadrada de amplitud A y período T:
\begin{enumerate}[label=\alph*)]
    \item Calcule analíticamente los coeficientes de la Serie de Fourier
    \item Escriba la expresión de la serie hasta el 5° armónico
    \item Explique el fenómeno de Gibbs
\end{enumerate}

\textbf{3.3)} Establezca la relación entre Series de Fourier y Transformada de Fourier. ¿Cuándo se utiliza cada una? ¿A que tipo de señales se aplican? ¿Cómo son estas señales desde el punto de vista de potencia y energía?

\textbf{3.4)} Enuncie y explique las siguientes propiedades de la Transformada de Fourier:
\begin{itemize}
    \item Linealidad
    \item Desplazamiento temporal
    \item Desplazamiento frecuencial
    \item Escalado
    \item Dualidad
    \item Convolución
    \item Parseval
\end{itemize}