\subsection*{4. Transformada de Hilbert y Señal Analítica}

\noindent \textbf{4.1)} Defina la Transformada de Hilbert de una señal \(x(t)\). ¿Cuál es su interpretación física?
\bigskip

La Transformada de Hilbert es una herramienta matemática utilizada para describir la envolvente compleja de una señal modulada con una portadora real. Se denota como \( \hat{x}(t) \) y se define mediante la integral:

\[
\hat{x}(t) = \mathcal{H}\{x(t)\} = \frac{1}{\pi} \int_{-\infty}^{\infty} \frac{x(\tau)}{t - \tau}  d\tau
\]

Equivalentemente, es la convolución de \( x(t) \) con la función \( \frac{1}{\pi t} \):

\[
\hat{x}(t) = x(t) * \frac{1}{\pi t}
\]

\noindent \textbf{Interpretación Física:} \par
    \noindent La Transformada de Hilbert desplaza la fase de todas las componentes frecuenciales de la señal en \(-90^\circ\) (es decir, \( -\frac{\pi}{2} rad\)), no afecta la amplitud del espectro, solo modifica la fase. Puede verse como un filtro pasa-todo con respuesta en frecuencia:
    \[
    H(f) = -j \cdot \operatorname{sgn}(f) = 
    \begin{cases}
        -j & \text{para } f > 0 \\
        j & \text{para } f < 0
    \end{cases}
    \]

    \noindent Esto significa que para frecuencias positivas desplaza la fase \(-90^\circ\) y para frecuencias negativas desplaza la fase \(+90^\circ\).\par
    \bigskip

%--------------------------------------------------------------------------------

\noindent \textbf{4.2)} Explique el concepto de señal analítica. ¿Cuáles son sus ventajas en el procesamiento de señales?\par
\bigskip

\noindent La señal analítica es una representación compleja de una señal real, obtenida mediante la supresión de las componentes frecuenciales negativas de la señal original. A partir de la señal analítica se deriva la señal banda base equivalente, representación también analítica conocida como envolvente compleja.Se define como:
                        \[
                        x_a(t) = x(t) + j\hat{x}(t)
                        \]
                        
donde:
\begin{itemize}
    \item \( x(t) \) es la señal real original
    \item \( \hat{x}(t) \) es su transformada de Hilbert
\end{itemize}

\noindent \textbf{Ventajas:}
\begin{itemize}
    \item Reducción del ancho de banda: al eliminar las frecuencias negativas, el ancho de banda efectivo se reduce a la mitad.
    \item Facilita el análisis de envolventes y fases: la señal analítica permite definir fácilmente la envolvente compleja y la fase instantánea de una señal modulada.
    \item Modulación de Banda Lateral Única (SSB).
    \item Demodulación coherente.
    \item Análisis de señales pasabanda.
\end{itemize}
%--------------------------------------------------------------------------------
\bigskip 

\noindent \textbf{4.3)} Si \(x(t) = \cos(\omega t)\), encuentre:\par
\bigskip

\noindent a) Su transformada de Hilbert.\par
\[
\hat{x}(t) = sin(\omega t)
\]

\noindent b) La señal analítica correspondiente. \par
\[
x_a(t) = x(t) + j \hat{x}(t) = \cos(\omega t) + j \sin(\omega t)
\]
Usando la identidad de Euler:
\[
e^{j\theta} = \cos\theta + j\sin\theta,
\]
con \(\theta = \omega t\), resulta:
\[
x_a(t) = e^{j\omega t}.
\]


\noindent c) La envolvente compleja.\par
\[
\tilde{x}(t) = x_a(t) \cdot e^{-j\omega t} = e^{j\omega t} \cdot e^{-j\omega t} = e^{0} = 1.
\]
